\documentclass[12pt]{article}
\usepackage{amsmath,amssymb,amsthm,geometry}
\usepackage[colorlinks=true, linkcolor=blue, citecolor=red, urlcolor=magenta]{hyperref}
\geometry{a4paper, margin=1in}

\title{Foundations of Logic}
\author{Cemil Derman}
\date{}

\theoremstyle{definition}
\newtheorem{definition}{Definition}[section]

\begin{document}
\maketitle

\section{Preliminaries}

\begin{definition}[Logic]
"Is the discipline that studies the principles of valid reasoning", philosopher might say.
"Is the study of consequence" mathematician might say,
Created in late 19th century and used in the proof of things that we can't really comprehend.
  Logic studies the contradictions, the paths which proof may lead, quality of given axioms and the end result. 

\end{definition}

\begin{definition}[Proof]
       Consists of axioms/premises and conclusion. axioms/premises: thing that taken as truth. 
        $InferenceRules(Axioms) \rightarrow NewTrueStatement$ and a new theorem is popped out of thin air -which we assumed air exists-.
    If given axioms doesn't reach desired conclusion then given proposition is bad. 

\end{definition}
\begin{definition}[Logic and Proof]
  Studies whether given axioms in the proof is good or bad,
  is the proof logically true or not, does it makes sense in the mathematical view?
  
\end{definition}

The basis of Logic and Proof is \href{../tools/logicandproof.pdf}{Propositional Logic}.
After that First-order Logic comes.
And after first-order logic, you don't need to get deeper than that but theres second-order logic, type theory (if you like cs), higher-order(nth, third)
\end{document}

