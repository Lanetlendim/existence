%%%%%%%%%%%%%%%%%%%%%%%%%%%%% Define Article %%%%%%%%%%%%%%%%%%%%%%%%%%%%%%%%%%
\documentclass{article}
%%%%%%%%%%%%%%%%%%%%%%%%%%%%%%%%%%%%%%%%%%%%%%%%%%%%%%%%%%%%%%%%%%%%%%%%%%%%%%%

%%%%%%%%%%%%%%%%%%%%%%%%%%%%% Using Packages %%%%%%%%%%%%%%%%%%%%%%%%%%%%%%%%%%
\usepackage{geometry}
\usepackage{graphicx}
\usepackage{amssymb}
\usepackage{amsmath}
\usepackage{amsthm}
\usepackage{empheq}
\usepackage{mdframed}
\usepackage{booktabs}
\usepackage{lipsum}
\usepackage{graphicx}
\usepackage{color}
\usepackage{psfrag}
\usepackage{pgfplots}
\usepackage{bm}
\usepackage{hyperref}
%%%%%%%%%%%%%%%%%%%%%%%%%%%%%%%%%%%%%%%%%%%%%%%%%%%%%%%%%%%%%%%%%%%%%%%%%%%%%%%

% Other Settings

%%%%%%%%%%%%%%%%%%%%%%%%%% Page Setting %%%%%%%%%%%%%%%%%%%%%%%%%%%%%%%%%%%%%%%
\geometry{a4paper}

%%%%%%%%%%%%%%%%%%%%%%%%%% Define some useful colors %%%%%%%%%%%%%%%%%%%%%%%%%%
\definecolor{ocre}{RGB}{243,102,25}
\definecolor{mygray}{RGB}{243,243,244}
\definecolor{deepGreen}{RGB}{26,111,0}
\definecolor{shallowGreen}{RGB}{235,255,255}
\definecolor{deepBlue}{RGB}{61,124,222}
\definecolor{shallowBlue}{RGB}{235,249,255}
%%%%%%%%%%%%%%%%%%%%%%%%%%%%%%%%%%%%%%%%%%%%%%%%%%%%%%%%%%%%%%%%%%%%%%%%%%%%%%%

%%%%%%%%%%%%%%%%%%%%%%%%%% Define an orangebox command %%%%%%%%%%%%%%%%%%%%%%%%
\newcommand\orangebox[1]{\fcolorbox{ocre}{mygray}{\hspace{1em}#1\hspace{1em}}}
%%%%%%%%%%%%%%%%%%%%%%%%%%%%%%%%%%%%%%%%%%%%%%%%%%%%%%%%%%%%%%%%%%%%%%%%%%%%%%%

%%%%%%%%%%%%%%%%%%%%%%%%%%%% English Environments %%%%%%%%%%%%%%%%%%%%%%%%%%%%%
\newtheoremstyle{mytheoremstyle}{3pt}{3pt}{\normalfont}{0cm}{\rmfamily\bfseries}{}{1em}{{\color{black}\thmname{#1}~\thmnumber{#2}}\thmnote{\,--\,#3}}
\newtheoremstyle{myproblemstyle}{3pt}{3pt}{\normalfont}{0cm}{\rmfamily\bfseries}{}{1em}{{\color{black}\thmname{#1}~\thmnumber{#2}}\thmnote{\,--\,#3}}
\theoremstyle{mytheoremstyle}
\newmdtheoremenv[linewidth=1pt,backgroundcolor=shallowGreen,linecolor=deepGreen,leftmargin=0pt,innerleftmargin=20pt,innerrightmargin=20pt,]{theorem}{Theorem}[section]
\theoremstyle{mytheoremstyle}
\newmdtheoremenv[linewidth=1pt,backgroundcolor=shallowBlue,linecolor=deepBlue,leftmargin=0pt,innerleftmargin=20pt,innerrightmargin=20pt,]{definition}{Definition}[section]
\theoremstyle{myproblemstyle}
\newmdtheoremenv[linecolor=black,leftmargin=0pt,innerleftmargin=10pt,innerrightmargin=10pt,]{problem}{Problem}[section]
%%%%%%%%%%%%%%%%%%%%%%%%%%%%%%%%%%%%%%%%%%%%%%%%%%%%%%%%%%%%%%%%%%%%%%%%%%%%%%%

%%%%%%%%%%%%%%%%%%%%%%%%%%%%%%% Plotting Settings %%%%%%%%%%%%%%%%%%%%%%%%%%%%%
\usepgfplotslibrary{colorbrewer}
\pgfplotsset{width=8cm,compat=1.9}
%%%%%%%%%%%%%%%%%%%%%%%%%%%%%%%%%%%%%%%%%%%%%%%%%%%%%%%%%%%%%%%%%%%%%%%%%%%%%%%

%%%%%%%%%%%%%%%%%%%%%%%%%%%%%%% Title & Author %%%%%%%%%%%%%%%%%%%%%%%%%%%%%%%%
\title{Set notation}
\author{Samuel Dearman}
%%%%%%%%%%%%%%%%%%%%%%%%%%%%%%%%%%%%%%%%%%%%%%%%%%%%%%%%%%%%%%%%%%%%%%%%%%%%%%%

\begin{document}
    \maketitle
    Set's generally expressed by an uppercased letter with some fancy style. \\
      $\mathbb{N}$ Natural Numbers, \\
      $\mathbb{Z}$ Integers, \\
      $\mathbb{Q}$ Rational Numbers, \\
      $\mathbb{I}$ Irrational Numbers, \\
      $\mathbb{R}$ Real Numbers, \\ 
      $\mathbb{C}$ Complex Numbers \\
      As in name, a set contains a collection of things. one might say id of set is its every element. \\ 
      let's create a set, named $\mathbb{S}$ as in Samuel. 
      What do we know of this set? only information we have is its name. But what's the use of that?
      We know we need more information for the sake of this god damn set theory to be useful. 
      Well, we said a set contains collection of things, 'elements'. Let's talk about that. \\
      $\mathbb{S}$ has 3 things i like. i like, uuhh, okay. 
      $\mathbb{S}$ contains 1 thing, my name. 
      and my name is n, $n="Samuel"$\\
      And i want to show that! 
      \[
      n \in \mathbb{S}
    \]
      okay, now we know how to show by finger, but thats rude. Now we do this;
     \[
       s="Dearman" \to 
      s \in \mathbb{S} 
      \]
      \footnote{\href{file:https://www.google.com}{What is that arrow?}} %% TODO: finish logical symbolism
      from what we know of, $\mathbb{S} = \{n, s, \dots\}$ \\
      So, it's something like my credentials
      okay fuck this shit ass story telling teaching. \\
      \begin{math}
        \mathbb{S} = \{n,s,\dots\} \\
        \mathbb{A} = \{n, s\} \implies \mathbb{A} \subseteq \mathbb{S} \\
        \text{(if elements of a set is in another set then one of them is a subset or they're equal)} \\ 
        \mathbb{A} \cap \mathbb{S} = \mathbb{A} \text{       }  \footnote{\href{}{A returning to itself is associated with identity law. it's examined in category theory more deeply. }} \\ %% TODO: Finish category theory about this part!
        \cap \text{ is a symbol that represents intersection.} \\ 
        \mathbb{B} \cap \mathbb{S} = \varnothing \\ 
        \text{means two sets have no common element.} \\ 
        \mathbb{A} \cup \mathbb{S} = \mathbb{S} \\ 
        \text{union of two sets,  (} \mathbb{B} \cup \mathbb{S} \neq  \mathbb{S} \text{)}\\
      \end{math}
\end{document}
