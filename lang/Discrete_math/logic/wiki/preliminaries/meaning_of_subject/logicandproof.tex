\documentclass[12pt]{article}
\usepackage{amsmath,amssymb,amsthm,geometry,hyperref}
\geometry{a4paper, margin=1in}

\title{Foundations of Logic}
\author{Cemil Derman}
\date{}

\theoremstyle{definition}
\newtheorem{definition}{Definition}[section]

\theoremstyle{plain}
\newtheorem{theorem}{Theorem}[section]

\begin{document}
\maketitle

\section{Preliminaries}

\begin{definition}[Logic]
"Is the discipline that studies the principles of valid reasoning", one might say.
  Logic studies the contradictions, the paths which proof may lead. In general cases it studies the consequences,
  One might be angry at the problem of which;
  "if 5 is even, then six is even" is a logically true sentence, although the connection within the sentence is not valid.

\end{definition}

\begin{definition}[Proof]
       Consists of axioms/premises and conclusion.
    If given axioms doesn't reach desired conclusion then given proposition is bad. 

\end{definition}
\begin{definition}[Logic and Proof]
  Studies whether given axioms in the proof is good or bad,
  is the proof logically true or not, does it makes sense in the mathematical view?
  
\end{definition}

\end{document}

